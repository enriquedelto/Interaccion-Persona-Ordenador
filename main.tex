\documentclass{report}
% Configuración
\input{config.tex}
\usepackage[a4paper]{geometry}
\usepackage{graphicx, wrapfig, subcaption, setspace, booktabs}
\usepackage[T1]{fontenc}
\usepackage[spanish]{babel}
\usepackage[scaled]{helvet}     % Fuente del documento
\renewcommand{\familydefault}{\sfdefault}
\usepackage[utf8]{inputenc}
\usepackage{url, lipsum}
\usepackage{tabularx}
\usepackage{multicol}
\usepackage{enumitem}
\usepackage{listings}
\usepackage{xcolor}

\lstset{
    basicstyle=\ttfamily\small,
    commentstyle=\color{gray},
    numbers=left,
    numberstyle=\tiny,
    frame=single,
    tabsize=4,
    breaklines=true,
    showstringspaces=false
}

% Puedes cambiar el color principal
% \definecolor{primary}{HTML}{}

\begin{document}

\pagestyle{empty}
\portada{NOTAS DE IPO}{Plantilla de notas}{Enrique M. Delgado Torres}
\newpage

\tableofcontents

\newpage

\chapter{Introducción a la Interacción Persona Ordenador}

\section{Concepto Interfaz en el contexto de la IPO}
La interfaz refleja las \textbf{Propiedades Físicas}, las \textbf{Funciones} y el \textbf{Balance de Poder y Control}, facilitando así la transmisión de información, órdenes y datos. Además, permite compartir sensaciones, intuiciones y nuevas formas de ver las cosas entre humanos y sistemas informáticos.

\vspace{4mm} % Añade un espacio vertical de 4mm entre los párrafos

\begin{itemize}[itemsep=0.1em] % Ajusta la separación de los ítems aquí
    \item \textbf{Propiedades Físicas:} La interfaz está diseñada teniendo en cuenta tanto las necesidades físicas y ergonómicas del usuario como las características del sistema informático, garantizando comodidad y accesibilidad.
    \item \textbf{Funciones:} La interfaz muestra y facilita las tareas específicas que el usuario necesita realizar, asegurando una interacción intuitiva y eficiente.
    \item \textbf{Balance de Poder y Control:} La interfaz define quién controla la interacción, ya sea el usuario o el sistema, influyendo en la experiencia del usuario y la eficacia del sistema.
\end{itemize}

\section{¿Qué es la IPO?}
La Interacción Persona-Ordenador (IPO) es una disciplina multidisciplinaria dedicada al diseño, evaluación e implementación de interfaces que faciliten una interacción entre seres humanos y sistemas informáticos de forma segura, efectiva, útil, eficiente, usable, accesible e inclusiva.

\vspace{4mm} % Añade un espacio vertical de 4mm entre los párrafos

La IPO se enfoca en la adaptabilidad, asegurando que las interfaces se desarrollen y evolucionen según las necesidades cambiantes de los usuarios y los avances tecnológicos. 

\vspace{4mm}

Traslada los conocimientos de diversas disciplinas para desarrollar herramientas y técnicas que ayuden a los diseñadores a crear sistemas informáticos idóneos.

\section{Disciplinas Integradas en la IPO}
La IPO combina conocimientos y métodos de diversas disciplinas, incluyendo:
\begin{multicols}{3}
\begin{itemize}[itemsep=0.1em] % Ajusta la separación de los ítems aquí
    \item Diseño
    \item Psicología
    \item Ergonomía
    \item Programación
    \item Ingeniería de Software
    \item Inteligencia Artificial
    \item Sociología
\end{itemize}
\end{multicols}

\subsection{Diseño}
Disciplina centrada en crear interfaces que son tanto funcionales como estéticamente agradables, buscando mejorar la experiencia y el entorno del usuario mediante soluciones intuitivas, accesibles y atractivas.

\subsection{Psicología}
Disciplina centrada en estudiar cómo los individuos y grupos procesan información, se comportan y toman decisiones al interactuar con sistemas informáticos. 

Cabe distinguir entre:
    \begin{itemize}
        \item \textbf{La Psicología Cognitiva} se enfoca en entender procesos mentales individuales, como percepción y toma de decisiones.
        \item \textbf{La Psicología Social} examina cómo el entorno social afecta el comportamiento del usuario. Ambas ramas contribuyen a diseñar interfaces más intuitivas, eficientes y satisfactorias para los usuarios.
    \end{itemize}

\subsection{Ergonomía}
Disciplina centrada en el diseño de interfaces y entornos de trabajo que maximizan la comodidad, eficiencia y seguridad. Esto incluye la organización óptima de controles y pantallas, consideración de factores físicos como la iluminación y la posición, y el uso adecuado de colores para facilitar la interacción y prevenir riesgos de salud.

\subsection{Programación}
La programación es el proceso de diseñar y codificar programas de computadora para resolver problemas o realizar tareas específicas. Se fundamenta en diversos paradigmas, cada uno con su enfoque y metodología distintiva:

\begin{itemize}
\item \textbf{Programación Orientada a Objetos:} Emplea clases y objetos para estructurar el código. Facilita la modularidad y reutilización. 

\textit{Ejemplo: Java, Python.}

\begin{minipage}{\linewidth}
    \begin{lstlisting}[language=Java, caption=Ejemplo de POO en Java]
    // Definicion de la clase "Coche"
    // Una clase es una plantilla para crear objetos
    public class Coche {
        // Atributos de la clase
        // Representan las caracteristicas del objeto
        private String marca;
        private int ano;
    
        // Constructor de la clase
        // Inicializa un nuevo objeto de la clase Coche
        public Coche(String marca, int ano) {
            this.marca = marca;
            this.ano = ano;
        }
    
        // Metodos de la clase
        // Definen el comportamiento del objeto
        public String getMarca() {
            return marca;
        }
    
        public int getAno() {
            return ano;
        }
    }
    
    // Clase principal para ejecutar el programa
    public class Main {
        public static void main(String[] args) {
            // Creacion de un objeto "Coche"
            // Un objeto es una instancia de una clase
            Coche miCoche = new Coche("Toyota", 2021);
    
            // Uso de metodos del objeto
            System.out.println("Marca del Coche: " + miCoche.getMarca());
            System.out.println("Ano del Coche: " + miCoche.getAno());
        }
    }
    \end{lstlisting}
    \end{minipage}

\item \textbf{Programación Imperativa:} Centrada en la secuencia de comandos para manipular el estado de las variables. 

\textit{Ejemplo: C, Pascal.}

\begin{minipage}{\linewidth} % Empieza el entorno minipage
\begin{lstlisting}[language=C, caption=Ejemplo de Programacion Imperativa en C]
#include <stdio.h>

int main() {
    int contador = 0; // Inicializa una variable

    // Ciclo para incrementar la variable
    for (int i = 0; i < 5; i++) {
        contador += i;
        printf("Valor actual del contador: %d\n", contador);
    }

    return 0;
}
\end{lstlisting}

% Añadir la salida esperada del código
\textbf{Salida esperada:}
\begin{verbatim}
Valor actual del contador: 0
Valor actual del contador: 1
Valor actual del contador: 3
Valor actual del contador: 6
Valor actual del contador: 10
\end{verbatim}
\end{minipage} % Termina el entorno minipage

\item \textbf{Programación Funcional:} Aborda los problemas mediante funciones, enfocándose en las relaciones de entrada y salida. 

\textit{Ejemplo: Haskell, Lisp.}

\begin{minipage}{\linewidth} % Empieza el entorno minipage
\begin{lstlisting}[language=Haskell, caption=Ejemplo de Programacion Funcional en Haskell]
-- Define una funcion simple que duplica un numero
duplicar :: Int -> Int
duplicar x = x * 2

-- Funcion principal
main :: IO ()
main = print (duplicar 5)
\end{lstlisting}

% Añadir la salida esperada del código
\textbf{Salida esperada:}
\begin{verbatim}
10
\end{verbatim}
\end{minipage} % Termina el entorno minipage

\item \textbf{Programación Declarativa:} Se concentra en describir el \textit{qué} de las operaciones, dejando el \textit{cómo} a la interpretación del sistema. 

\textit{Ejemplo: SQL, Prolog.}

\begin{minipage}{\linewidth} % Empieza el entorno minipage
\begin{lstlisting}[language=SQL, caption=Ejemplo de Programacion Declarativa en SQL]
-- SQL para seleccionar nombres de una tabla de empleados
SELECT nombre FROM empleados WHERE edad > 30;
\end{lstlisting}

% Añadir la salida esperada del código
\textbf{Salida esperada:}
\begin{verbatim}
[Lista de nombres de empleados mayores de 30 años]
\end{verbatim}
\end{minipage} % Termina el entorno minipage

\item \textbf{Programación Concurrente:} Maneja operaciones que se ejecutan simultáneamente, esencial en aplicaciones multitarea y multiusuario. 

\textit{Ejemplo: Erlang, Go.}

\begin{minipage}{\linewidth} % Empieza el entorno minipage
\begin{lstlisting}[language=Go, caption=Ejemplo de Programacion Concurrente en Go]
package main

import (
    "fmt"
    "time"
)

// Funcion que se ejecutara de manera concurrente
func imprimirNumeros() {
    for i := 1; i <= 5; i++ {
        time.Sleep(1 * time.Second)
        fmt.Println(i)
    }
}

func main() {
    // Ejecutar la funcion imprimirNumeros de manera concurrente
    go imprimirNumeros()

    // Esperar a que el usuario presione una tecla
    fmt.Println("Presiona Enter para finalizar")
    fmt.Scanln()
}
\end{lstlisting}

\textbf{Salida esperada:}
\begin{verbatim}
Presiona Enter para finalizar
[Los números del 1 al 5 se imprimirán uno cada segundo]
\end{verbatim}
\end{minipage} % Termina el entorno minipage

\end{itemize}

\end{document}  